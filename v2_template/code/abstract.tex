\hspace{0pt}
\vfill

\section*{\centering Resumen}

El presente informe tiene como objetivo presentar el trabajo realizado para el cumplimiento de la asignatura Práctica Supervisada.
En primer lugar, se presentan los aspectos generales sobre compiladores y la producción de código intermedio. Luego, se pone el foco en el análisis estructural de las \textbf{suites} de compilación: GNU GCC y CLANG/LLVM.
Posteriormente, es detallado el proceso de generación y código resultante de cada etapa de los compiladores.
Luego, se procede a una breve explicación sobre los archivos ejecutables, resultantes del proceso de compilación. Más tarde, se hace foco en los archivos de tipo \texttt{ELF}: propiedades y estructura. 
Finalmente, se exponen los distintos métodos de inserción y posterior lectura de datos dentro de un archivo ejecutable de formato \texttt{ELF}.
\newline
Los programas de inserción y lectura se emplearán como base para el trabajo de Proyecto Integrador, siendo ambos un inicio para un mejor desarrollo en cuanto a profundidad, abstracción y uso en espacio de sistema (\emph{kernel}).

\vfill
\hspace{0pt}






