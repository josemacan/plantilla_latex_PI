%% ---- DOCUMENT CLASS
%% -------- Tamano hoja: A4, tamano fuente cuerpo texto: 11 pt, clase de documento: book, formato hoja: doble faz (twoside) 
\documentclass[a4paper, 11pt, twoside]{book}

\raggedbottom

%% ---- ARCHIVO DE ESTILOS
%% -------- Resto de configuraciones de formato se define en el archivo *formato_PI.sty*
\usepackage{formato_PI}


%% ---- PARTES DEL DOCUMENTO

\begin{document}


    \frontmatter

    \pagenumbering{roman}

    %--------------------------------------------------------------------------------------------------------------------------
    %  PORTADA
    %--------------------------------------------------------------------------------------------------------------------------
    \begin{titlepage}
    \centering
    {\includegraphics[width=0.225\textwidth]{../img/logo.jpg}}
    \hspace{1cm}
    {\includegraphics[width=0.15\textwidth]{../img/fcefyn.png}\par}
    \vspace{1cm}
    {\bfseries\LARGE Universidad Nacional de Córdoba \par}
    \vspace{0.7cm}
    {\scshape\Large Facultad de Ciencias Exactas, Físicas y Naturales\par}
    \vspace{2cm}
    {\scshape\Huge Compiladores y ejecutables \par}
    \vspace{2cm}
    {\itshape\Large Práctica Supervisada \par}
    \vspace{1.5cm}
    {\itshape\large Informe de Trabajo \par}
    \vfill
    \vspace{0.5cm}
    {\Large Supervisor: \par}
    \vspace{0.3cm}
    {\Large \textbf{Prof. Maximiliano Eschoyez}\par}
    \vspace{0.5cm}
    {\Large Tutor: \par}
    \vspace{0.3cm}
    {\Large \textbf{Nicolás Papp}\par}
    \vspace{0.5cm}
    {\Large Autores: \par}
    \vspace{0.3cm}
    {\Large \textbf{\emph{José Cancinos}} y \textbf{\emph{Julián González}}\par}
    \vfill
    \vspace{0.5cm} 
    {\small 2021 \par}
\end{titlepage}



    %--------------------------------------------------------------------------------------------------------------------------
    %  ACTA DE EXAMEN
    %--------------------------------------------------------------------------------------------------------------------------
    \chapter*{}
\begin{center}
	
	% Logo universidad
	\begin{figure}[h]
		\begin{center}
			\includegraphics[scale=0.8]{logo_unc.png}
		\end{center}
	\end{figure}
	\vspace{0.25em}
	
	\textsc{\LARGE Universidad Nacional de Córdoba}\\[0.3cm] % Name of your university/college
	\textsc{\large Facultad de Ciencias Exactas, Físicas y Naturales}\\[0.3cm] % Major heading such as course name
	\textsc{\large Escuela de Ingeniería en Computación}\\[0.75cm] % Minor heading such as course title
\end{center}


El Tribunal Evaluador reunido en este acto y luego de haber aprobado la Solicitud de Aprobación de Tema 
y efectuado las distintas instancias de correcciones del Informe del Proyecto Integrador para la obtención 
del Título de Grado “Ingeniero Electrónico” y cumpliendo con el Reglamento correspondiente, declaran el Informe 
Final de los estudiantes \textbf{Apellido, Nombres} y \textbf{Apellido, Nombres} como “aceptado sin correcciones” 
y la defensa oral Aprobada. Por lo tanto, luego de haber tenido en cuenta los aspectos de evaluación que indica el 
Reglamento, el Proyecto Integrador se considera Aprobado. 

\vspace{0.5cm}

\par Se firma el Acta de Examen correspondiente y se distribuyen los ejemplares impresos.

\vspace{4.5cm}

Firma y aclaración del Tribunal Evaluador:

\vspace{0.5cm}

Fecha:

\clearpage{\thispagestyle{empty}\cleardoublepage}       %% Comando para insertar pagina en blanco al ser doble pagina y remover fancy headers

    %--------------------------------------------------------------------------------------------------------------------------
    %  DEDICATORIA
    %--------------------------------------------------------------------------------------------------------------------------
    \chapter*{Dedicatoria}
\addcontentsline{toc}{chapter}{Dedicatoria}

\begin{flushright}
	\textit{Para nuestras familias...}
\end{flushright}

\clearpage{\thispagestyle{empty}\cleardoublepage}       %% Comando para insertar pagina en blanco al ser doble pagina y remover fancy headers

    %--------------------------------------------------------------------------------------------------------------------------
    %  AGRADECIMIENTOS
    %--------------------------------------------------------------------------------------------------------------------------
    \chapter*{Agradecimientos}
\addcontentsline{toc}{chapter}{Agradecimientos} 

\textit{A nuestros padres, madres y hermanos, por su incondicional apoyo a lo largo de toda la carrera.}

\vspace{1cm}

\textit{A nuestros directores, Apellido Nombre y Apellido Nombre, por la excelente predisposición, la confianza y todo el soporte brindado que hizo posible este proyecto.}

\vspace{1cm}

\textit{A nuestros amigos y futuros colegas, quienes hicieron de estos años de estudio una experiencia más placentera.}

\vspace{1cm}

\textit{A -Lugar donde se realizó el PI-, junto a todo su personal, por las oportunidades y enseñanzas compartidas.}

\vspace{1cm}

\textit{A la Facultad de Ciencias Exactas, Físicas y Naturales de la Universidad Nacional de Córdoba, por la oportunidad de realizar esta carrera de grado.}


\clearpage{\thispagestyle{empty}\cleardoublepage}       %% Comando para insertar pagina en blanco al ser doble pagina y remover fancy headers

    %--------------------------------------------------------------------------------------------------------------------------
    %  RESUMEN
    %--------------------------------------------------------------------------------------------------------------------------
    \chapter*{Resumen}
\addcontentsline{toc}{chapter}{Resumen}

En este trabajo...

%% -------------------------------------------
\vspace{4.5cm}  %% BORRAR AL COMPLETAR SECCION
%% -------------------------------------------

\textbf{Áreas Temáticas del Proyecto Integrador}: 

\textbf{Asignaturas}: 

\textbf{Palabras Claves}:

\clearpage{\thispagestyle{empty}\cleardoublepage}       %% Comando para insertar pagina en blanco al ser doble pagina y remover fancy headers

    %--------------------------------------------------------------------------------------------------------------------------
    %  ABSTRACT
    %--------------------------------------------------------------------------------------------------------------------------
    \hspace{0pt}
\vfill

\section*{\centering Resumen}

El presente informe tiene como objetivo presentar el trabajo realizado para el cumplimiento de la asignatura Práctica Supervisada.
En primer lugar, se presentan los aspectos generales sobre compiladores y la producción de código intermedio. Luego, se pone el foco en el análisis estructural de las \textbf{suites} de compilación: GNU GCC y CLANG/LLVM.
Posteriormente, es detallado el proceso de generación y código resultante de cada etapa de los compiladores.
Luego, se procede a una breve explicación sobre los archivos ejecutables, resultantes del proceso de compilación. Más tarde, se hace foco en los archivos de tipo \texttt{ELF}: propiedades y estructura. 
Finalmente, se exponen los distintos métodos de inserción y posterior lectura de datos dentro de un archivo ejecutable de formato \texttt{ELF}.
\newline
Los programas de inserción y lectura se emplearán como base para el trabajo de Proyecto Integrador, siendo ambos un inicio para un mejor desarrollo en cuanto a profundidad, abstracción y uso en espacio de sistema (\emph{kernel}).

\vfill
\hspace{0pt}






    

    %--------------------------------------------------------------------------------------------------------------------------
    % ÍNDICE
    %--------------------------------------------------------------------------------------------------------------------------
    \addcontentsline{toc}{chapter}{Índice}

\tableofcontents




    %--------------------------------------------------------------------------------------------------------------------------
    % LISTA DE FIGURAS
    % Incluir numero de figura, titulo de la figura y numero de pagina.
    %--------------------------------------------------------------------------------------------------------------------------
    \chapter*{Lista de Figuras}
\addcontentsline{toc}{chapter}{Lista de Figuras}

\listoffigures
\clearpage{\thispagestyle{empty}\cleardoublepage}       %% Comando para insertar pagina en blanco al ser doble pagina y remover fancy headers

    %--------------------------------------------------------------------------------------------------------------------------
    % LISTA DE TABLAS
    % Incluir numero de tabla, titulo de la tabla y numero de pagina.
    %--------------------------------------------------------------------------------------------------------------------------
    \chapter*{Lista de Tablas}
\addcontentsline{toc}{chapter}{Lista de Tablas}

\listoftables{}

\clearpage{\thispagestyle{empty}\cleardoublepage}       %% Comando para insertar pagina en blanco al ser doble pagina y remover fancy headers

    %--------------------------------------------------------------------------------------------------------------------------
    % LISTA DE CODIGOS
    % Incluir numero de listing de codigo, titulo de la listing y numero de pagina.
    %--------------------------------------------------------------------------------------------------------------------------
    \addcontentsline{toc}{chapter}{Lista de Códigos}
\renewcommand\lstlistlistingname{Lista de Códigos}

\lstlistoflistings{}

\clearpage{\thispagestyle{empty}\cleardoublepage}       %% Comando para insertar pagina en blanco al ser doble pagina y remover fancy headers

    %--------------------------------------------------------------------------------------------------------------------------
    % LISTA DE ACRÓNIMOS
    %--------------------------------------------------------------------------------------------------------------------------
    \chapter*{Lista de Símbolos y Convenciones}
\addcontentsline{toc}{chapter}{Lista de Símbolos y Convenciones}

\begin{acronym}

    \acro{GCC}{GNU Compiler Collection}

\end{acronym}

\clearpage{\thispagestyle{empty}\cleardoublepage}       %% Comando para insertar pagina en blanco al ser doble pagina y remover fancy headers
 

    %------------------
    \mainmatter
    %------------------

        \pagenumbering{arabic}

        %--------------------------------------------------------------------------------------------------------------------------
        % CAPITULO 1: COMPILADORES
        %--------------------------------------------------------------------------------------------------------------------------
        \chapter{Introducción}

Párrafo con la descripción del contenido, lo que espera encontrar el lector.

\section*{Introducción}

Presentación general clara y breve del contenido del PI, no debe incluir resultados ni conclusiones. 
La introducción es lo primero que se lee por lo tanto se debe tener un especial cuidado en la redacción. 
Incluir éstos títulos:

\begin{itemize}
    \setstretch{1}
    \item Antecedentes breves del problema.
    \item Relevancia de trabajo.
    \item Motivación para la elección del tema. 
    \item Formulación del problema.
    \item Objetivo General y Objetivos Específicos.
    \item Metodología utilizada para lograr los objetivos propuestos.
    \item Orientación al lector de la organización del texto.
\end{itemize}

Los Capítulos se numeran del 1 (Introducción) al último en forma consecutiva incluyendo Marco Teórico y Marco Metodológico

% --------------------------------------------------------------------------------------------------
% --------------------------- SECCION: Antecedentes breves del problema  ---------------------------

\section{Antecedentes breves del problema}

    \textbf{BLABLABLA}: blablabla. \cite{von_hagen_definitive_2006}

% ---------------------------------------------------------------------------------------------------------------------------
% --------------------------- SECCION: Relevancia de trabajo  ---------------------------------------------------------------

\section{Relevancia de trabajo}

% ---------------------------------------------------------------------------------------------------------------------------
% --------------------------- SECCION: Motivación para la elección del tema  ------------------------------------------------

\section{Motivación para la elección del tema}

% ----------------------------------------------------------------------------------------------------------------------------
% --------------------------- SECCION: Formulación del problema  -------------------------------------------------------------

\section{Formulación del problema}

% ----------------------------------------------------------------------------------------------------------------------------
% --------------------------- SECCION: Objetivo General y Objetivos Específicos  ---------------------------------------------

\section{Objetivo General y Objetivos Específicos}

% ----------------------------------------------------------------------------------------------------------------------------
% --------------------------- SECCION: Metodología utilizada para lograr los objetivos propuestos  ---------------------------

\section{Metodología utilizada para lograr los objetivos propuestos}

% ----------------------------------------------------------------------------------------------------------------------------
% --------------------------- SECCION: Orientación al lector de la organización del texto  -----------------------------------

\section{Orientación al lector de la organización del texto}

        % ************************************************************************************************************************
        %                                       PARTE: MARCO TEORICO
        % ************************************************************************************************************************

        \part{Marco teórico}

        % El Marco Teórico puede tener varios Capítulos
        % Aquí se hace el desarrollo de los temas teóricos y conceptuales más importantes que impactan en el 
        % trabajo o que son específicos del mismo. No debe ser una recopilación de artículos y capítulos 
        % totales o parciales de libros. 
        % Debe ser un desarrollo del alumno en cuanto a lo teórico, basándose en la bibliografía o de 
        % temas vistos en clase. Debe incluir términos específicos, conceptos y teorías que ayuden a conocer 
        % la temática seleccionada
        % MUY IMPORTANTE: NO SE DEBE HACER REFERENCIA AL MARCO METODOLÓGICO. SI PUEDE TENER REFERENCIAS A LOS ANEXOS

        %--------------------------------------------------------------------------------------------------------------------------
        % CAPITULO 2: **CAPITULO 2**
        %--------------------------------------------------------------------------------------------------------------------------
        \chapter{Capítulo 2}

Un párrafo con la descripción del contenido, lo que espera encontrar el lector       


        % ************************************************************************************************************************
        %                                       PARTE: MARCO METODOLOGICO
        % ************************************************************************************************************************        

        \part{Marco metodológico}

        % El Marco Metodológico puede tener varios Capítulos
        % Aquí se desarrolla concretamente el PI, qué se hizo, cómo se hizo y con que se hizo. El procedimiento 
        % de diseño, simulación, armado de partes, ensambles, mediciones, etc.
        % Es conveniente que haga referencias al Marco Teórico y a los Anexos
        

        %--------------------------------------------------------------------------------------------------------------------------
        % CAPITULO N: COMPILADORES
        %--------------------------------------------------------------------------------------------------------------------------
        \include{code/capN-compiladores}    

        %--------------------------------------------------------------------------------------------------------------------------
        % CAPITULO N: DESCRIPCION DEL MODELO EXPERIMENTAL (OPCIONAL)
        %--------------------------------------------------------------------------------------------------------------------------
        \chapter{Descripción del modelo experimental}

ES UN CAPITULO OPCIONAL

Puede formar parte del último Capítulo o en un Capítulo aparte. 
Se describe el modelo terminado, su funcionalidad, operación, manual de instrucciones, mediciones, etc.
    
        
        %--------------------------------------------------------------------------------------------------------------------------
        % CAPITULO N: RESULTADOS
        %--------------------------------------------------------------------------------------------------------------------------
        \chapter{Resultados}

Aquí se describen los resultados obtenidos luego de todo el proceso a modo de resumen haciendo 
referencia a lo desarrollado anteriormente. 

Los Resultados no deben incluirse en un Capítulo, es un apartado individual.    

        %--------------------------------------------------------------------------------------------------------------------------
        % CAPITULO N: CONCLUSIONES
        %--------------------------------------------------------------------------------------------------------------------------
        \chapter{Conclusiones }

Las Conclusiones deben tener una redacción clara, concreta y directa. No son un resumen del trabajo.
Deben reflejar los alcances y las limitaciones del estudio, recomendaciones, indicaciones de posible continuidad del trabajo, etc.
Se sugieren incluir los aspectos siguientes:

\begin{itemize}
    \setstretch{1}
    \item Resultados obtenidos en relación a la Solicitud de Aprobación de Tema.
    \item Conclusión General.
    \item Aporte que hace a la Ingeniería o a un campo de conocimiento. 
\end{itemize}

Las Conclusiones no deben incluirse en un último Capítulo, es un apartado individual 



        % ----------------------
        % --------- BIBLIOGRAFÍA
        % ----------------------

        \begingroup     

            \nocite{*}                      % Imprimir todas las bibliografias y referencias, aunque no se hayan citado

            \setlength\bibitemsep{1.5pt}    % 1.5pt interlineado entre entradas
            \setstretch{1}                  % 1pt interlineado entre lineas

            \printbibliography[type=book, title=Bibliografía]
            \printbibliography[nottype=book, title=Referencias]

        \endgroup

        % ----------------------
        % --------- ANEXO
        % ----------------------

        \begin{appendices} 

            \renewcommand{\thechapter}{\Roman{chapter}}         % Numeracion de los anexos en numeros romanos

            %--------------------------------------------------------------------------------------------------------------------------
            % ANEXO 1: TITULO ANEXO 1 
            %--------------------------------------------------------------------------------------------------------------------------
            \chapter{Hello}

Puede haber varios Anexos, numerar e iniciar cada uno en hoja nueva

Todo material complementario se ordena en los anexos, según el tipo de material pueden utilizarse varios anexos. 
Ejemplos: 
Hojas de datos. Desarrollos complejos que se llevan al anexo para simplificar la lectura del 
cuerpo del trabajo. Procedimientos complementarios que son se incidencia directa en el trabajo. 
Líneas de código que no se incluyen en el texto para no complicar la lectura (si el código es 
extenso debe incluirse en un archivo en el DVD

     

            %--------------------------------------------------------------------------------------------------------------------------
            % ANEXO 2: TITULO ANEXO 2 
            %--------------------------------------------------------------------------------------------------------------------------
            \include{code/anexo2} 
    
        \end{appendices}


    \backmatter

\end{document}