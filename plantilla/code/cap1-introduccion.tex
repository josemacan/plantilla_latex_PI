\chapter{Introducción}

Párrafo con la descripción del contenido, lo que espera encontrar el lector.

\section*{Introducción}

Presentación general clara y breve del contenido del PI, no debe incluir resultados ni conclusiones. 
La introducción es lo primero que se lee por lo tanto se debe tener un especial cuidado en la redacción. 
Incluir éstos títulos:

\begin{itemize}
    \setstretch{1}
    \item Antecedentes breves del problema.
    \item Relevancia de trabajo.
    \item Motivación para la elección del tema. 
    \item Formulación del problema.
    \item Objetivo General y Objetivos Específicos.
    \item Metodología utilizada para lograr los objetivos propuestos.
    \item Orientación al lector de la organización del texto.
\end{itemize}

Los Capítulos se numeran del 1 (Introducción) al último en forma consecutiva incluyendo Marco Teórico y Marco Metodológico

% --------------------------------------------------------------------------------------------------
% --------------------------- SECCION: Antecedentes breves del problema  ---------------------------

\section{Antecedentes breves del problema}

    \textbf{BLABLABLA}: blablabla. \cite{von_hagen_definitive_2006}

% ---------------------------------------------------------------------------------------------------------------------------
% --------------------------- SECCION: Relevancia de trabajo  ---------------------------------------------------------------

\section{Relevancia de trabajo}

% ---------------------------------------------------------------------------------------------------------------------------
% --------------------------- SECCION: Motivación para la elección del tema  ------------------------------------------------

\section{Motivación para la elección del tema}

% ----------------------------------------------------------------------------------------------------------------------------
% --------------------------- SECCION: Formulación del problema  -------------------------------------------------------------

\section{Formulación del problema}

% ----------------------------------------------------------------------------------------------------------------------------
% --------------------------- SECCION: Objetivo General y Objetivos Específicos  ---------------------------------------------

\section{Objetivo General y Objetivos Específicos}

% ----------------------------------------------------------------------------------------------------------------------------
% --------------------------- SECCION: Metodología utilizada para lograr los objetivos propuestos  ---------------------------

\section{Metodología utilizada para lograr los objetivos propuestos}

% ----------------------------------------------------------------------------------------------------------------------------
% --------------------------- SECCION: Orientación al lector de la organización del texto  -----------------------------------

\section{Orientación al lector de la organización del texto}